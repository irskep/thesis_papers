\section{Metabolic Pathways}
\label{sect:metabolic_pathways}

\begin{verse}
    The information in this section comes from ``Artificial Intelligence and
    Molecular Biology'' by Lawrence E. Hunter \cite{mb-for-cs}.
\end{verse}

\textbf{This section still needs to be paraphrased. I have temporarily pasted
section 5.4 from the book here.}

The translation of genes into proteins, crucial as it is, is only a small
portion of the biochemical activity in a cell. Proteins do most of the work of
managing the flow of energy, synthesizing, degrading and transporting
materials, sending and receiving signals, exerting forces on the world, and
providing structural support. Systems of interacting proteins form the basis for
nearly every process of living things, from moving around and digesting food to
thinking and reproducing. Somewhat surprisingly, a large proportion of the
chemical processes that underlie all of these activities are shared across a
very wide range of organisms. These shared processes are collectively referred
to as intermediary metabolism. These include the catabolic processes for
breaking down proteins, fats and carbohydrates (such as those found in food) and
the anabolic processes for building new materials. Similar collections of
reactions that are more specialized to particular organisms are called secondary
metabolism. The substances that these reactions produce and consume are called
metabolites.

The biochemical processes in intermediary metabolism are almost all catalyzed
reactions. That is, these reactions would barely take place at all at normal
temperatures and pressures; they require special compounds that facilitate the
reaction — these compounds are called catalysts or enzymes. (It is only
partially in jest that many biochemistry courses open with the professor saying
that the reactions that take place in living systems are ones you were taught
were impossible in organic chemistry.) Catalysts are usually named after the
reaction they facilitate, usually with the added suffix -ase. For example,
alcohol dehydrogenase is the enzyme that turns ethyl alcohol into acetaldehyde
by removing two hydrogen atoms. Common classes of enzymes include
dehydrogenases, synthetases, proteases (for breaking down proteins),
decarboxylases (removing carbon atoms), transferases (moving a chemical group
from one place to another), kinases, phosphatases (adding or removing phosphate
groups, respectively) and so on. The materials transformed by catalysts are
called substrates. Unlike the substrates, catalysts themselves are not changed
by the reactions they participate in. A final point to note about enzymatic
reactions is that in many cases the reactions can proceed in either direction.
That is, and enzyme that transforms substance A into substance B can often also
facilitate the transformation of B into A. The direction of the transformation
depends on the concentrations of the substrates and on the energetics of the
reaction (see Mavrovouniotis’ chapter in this volume for further discussion of
this topic).

Even the basic transformations of intermediary metabolism can involve dozens or
hundreds of catalyzed reactions. These combinations of reactions, which
accomplish tasks like turning foods into useable energy or compounds are called
metabolic pathways. Because of the many steps in these pathways and the
widespread presence of direct and indirect feedback loops, they can exhibit many
counterintuitive behaviors. Also, all of these chemical reactions are going on
in parallel. Mavrovouniotis’s chapter in this volume describes an efficient
system for making inferences about these complex systems.

In addition to the feedback loops among the substrates in the pathways, the
presence or absence of substrates can affect the behavior of the enzymes
themselves, through what is called allosteric regulation. These interactions
occur when a substance binds to an enzyme someplace other than its usual active
site (the atoms in the molecule that have the enzymatic effect). Binding at
this other site changes the shape of the enzyme, thereby changing its activity.
Another method of controlling enzymes is called competitive inhibition. In
this form of regulation, substance other than the usual substrate of the enzyme
binds to the active site of the enzyme, preventing it from having an effect on
its substrate.

These are the basic mechanisms underlying eucaryotic cells (and much of this
applies to bacterial and archaeal ones as well). Of course, each particular
activity of a living system, from the capture of energy to immune response, has
its own complex network of biochemical reactions that provides the mechanism
underlying the function. Some of these mechanisms, such as the secondary
messenger system involving cyclic adenosine monophosphate (cAMP) are widely
shared by many different systems. Others are exquisitely specialized for a
particular task in a single species: my favorite example of this is the evidence
that perfect pitch in humans (being able to identify musical notes absolutely,
rather than relative to each other) is mediated by a single protein. The
functioning of these biochemical networks is being unravelled at an ever
increasing rate, and the need for sophisticated methods to analyze relevant data
and build suitable models is growing rapidly.
