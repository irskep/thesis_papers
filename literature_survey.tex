\emph{I did not edit this section at all since last time. It needs significant
editing and more information.}

Most literature related to these interfaces falls into two categories. The first
is general multitouch research, and the second is research on browsing and
querying graphs.

Most general multitouch research is not applicable to this specific area.
Conventions for basic tasks such as navigating between screens, browsing lists,
and scrolling content are well-defined by the interface guidelines of existing
touch-based operating systems. Any deviations from the beaten path are confined
to the graph view.

A few multitouch papers present taxonomies of gestures. Frisch et al provide one
example \cite{multitouch:gestures}. (I am looking for more.)

Others offer new interfaces to help bridge the divide between multitouch input
devices and the pre-iPad desktop interfaces. In this vein, Benko et al describe
a new interface to help a user select very small pointing targets
\cite{multitouch:tiny-select}. However, modern touch interfaces circumvent this
issue by not requiring users to select unreasonably tiny targets in the first
place.

Turning to graph interface research, Kobourov et al have a multi-touch interface
for making pairs of graphs with similarly placed nodes
\cite{graph-interaction:simultaneousgraphdrawing}, which will likely be
useful in the SMDA application. I have found one paper in which the
researchers use bubbles along the edges of the screen to represent offscreen
nodes (but I am so far unable to download it).

There is a lot of talk about UML, but it is not relevant to this application.
Another paper uses a hybrid graphical and text interface to explore graphs, but
text interfaces do not work well on touch-based devices.
