\section{SMDA Implementation}
\label{sect:smda_implementation}

The input interface for SMDA requires full knowledge of the SMDA database, so
\mawapp must store that database internally in some form.  It must store
information about pathways, metabolites, tissues, reactions, and more.

Section \ref{sect:smda_data_model} describes the subset of the SMDA data
model that is used by \mawapp to provide the SMDA tool. Section
\ref{sect:smda_web_services_server} enumerates the web services that
SMDA provides to download these model objects into \mawapp. Section
\ref{sect:smda_web_services_client} explains how \mawapp accesses these web
services and converts their results into model objects to be used by the
application. Finally, section \ref{sect:smda_results_ui} describes the user
interface classes.

\subsection{Data Model}
\label{sect:smda_data_model}

As was mentioned at the beginning of this section, \mawapp must store a
considerable subset of the SMDA database internally. These are the objects that
are stored from that database.

The Objective-C classes representing the model objects use the \texttt{SMDA}
prefix, but the prefix is omitted in this section for clarity.

\begin{itemize}

    \item \textbf{Pathway}: Information about a pathway including name,
    identifier, and the tissue it is contained in.

    \item \textbf{Transport Process}: Information about a transport process
    including the name and identifier.

    \item \textbf{Reaction}: Information about a reaction including name,
    identifier, tissue/compartment, whether the reaction is reversible, and
    metabolite pools the reaction is associated with.

    \item \textbf{Pathway as Reaction}: An entire pathway treated as if it were
    a reaction, in order to simplify the input and output for the user. Includes
    references to a pathway and metabolite pools that are inputs and outputs.

    \item \textbf{Tissue}: Information about a tissue/compartment including
    name, identifier, parent, and compartment type.

    \item \textbf{Metabolite}: Information about a metabolite including name,
    identifier, and metabolite pools it is associated with.

    \item \textbf{Metabolite Pool}: Information about a metabolite pool
    including name, identifier, tissue/compartment, role in its associated
    reaction, and the name and identifier of its associated metabolite.

\end{itemize}

In addition to objects directly from the SMDA database, the input includes
user-defined measurements. These are the objects used to model those
measurements.

\begin{itemize}

    \item \textbf{Metabolite Measurement}: The numeric value and units of
    measurement of a single metabolite pool.

    \item \textbf{Saved Measurement}: One full set of inputs for SMDA including
    pathways, transport processes, reactions, pathways as reactions,
    tissues/compartments, metabolites, and metabolite measurements.

\end{itemize}

\subsection{Loading the Data}
\label{sect:smda_all_data}

\subsubsection{Web Services}
\label{sect:smda_web_services}

This section describes the web services that provide the data used to populate
the models described in section \ref{sect:smda_data_model}.

Some web services simply list all items of a certain type from the database. The
services of this type that \mawapp uses are \textbf{GetAllPathways},
\textbf{GetAllUnits}, \textbf{GetAllTransportProcesses},
\textbf{GetAllReactions}, \textbf{GetAllCompartments},
\textbf{GetAllMetabolites}, and \textbf{GetAllMetabolitePools}. These web
services take no arguments and always return the same result until the database
is changed.

Other web services take a pathway as an argument in order to return results that
are relevant only to that pathway. The services of this type that \mawapp uses
are \textbf{GetTissueOfPathway}, which lists tissues/compartments, and
\textbf{GetPathwayMetabolite}, which lists metabolites.

The web service \textbf{GetAllMetabolitePoolsForPathway} takes both a pathway
and as a tissue and returns metabolite pools for that pathway-tissue pair.

The web service \textbf{GetReactionMetabolite} returns associations between
reactions and metabolite pools.

By calling these web services with all possible values, the entire database can
be mirrored on the iPad.

\emph{I did not have time to enumerate all of the web services' responses. It
will take about three pages.}

\subsubsection{Storage}

Due to the hierarchical and interconnected nature of the data model, it is not
practical to dynamically invoke the web services at the time their data is
needed. Instead, an internal representation is updated in bulk at the user's
request. When they tap an ``Update Local Data'' button in the SMDA input view, a
series of asynchronously executed Objective-C blocks (\ref{sect:objc_bocks}) are
invoked. These blocks make the web requests, convert their results to the model
objects described in section \ref{sect:smda_data_model}, and serialize them to
the iPad's internal storage.

Each block represents one or more web requests, and a block may rely on the
completion of another block for the existence of data.

To avoid delays from downloading all the data the first time the application is
run, the application binary contains a snapshot of these objects taken from the
web services at the time the application was released. This way, the application
does not require internet access until the actual SMDA algorithm is run.

\subsection{Getting SMDA Algorithm Results}
\label{sect:smda_results_request}

\emph{This section will describe how an SMDA web service request is formed,
submitted, read, parsed, and made into graphs. It will also explain \mawappp
inability to display results from multiple pathways due to the lack of dynamic
layouts.}

\subsection{User Interface}
\label{sect:smda_results_ui}

\emph{I did not have time to go over the SMDA input and output interface
implementations.}
