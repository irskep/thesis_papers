The \textbf{PathCase} System is an integrated software system for storing,
managing, analyzing, and querying biological pathways at different levels of
genetic, molecular, biochemical and organismal detail \cite{pathcase-basic}. At
the computational level, PathCase allows users to visualize pathways in multiple
abstraction levels, and to pose predetermined as well as ad hoc queries using a
graphical user interface. Pathways are represented as graphs, and implemented as
a relational database. PathCase has multiple levels, with multiple tools at each
level. Currently, users can access three different PathCase system applications,
each employing a different metabolic pathways database.

My thesis includes iPad-based interfaces to two of these databases. The first is
the Metabolomics Analysis Workbench, designed for metabolomics analysis.  The
PathCaseMAW database contains a fully hierarchically compartmentalized metabolic
network for humans and mice. Full compartment hierarchy refers to the
multi-tissue (e.g.  liver, adipose tissue, muscle, etc) environment as well as a
complete biological compartment distinction (e.g., liver-cell, liver-cytosol,
liver-mitochondrion, etc.).

The second is a database of KEGG pathways, intended more for students and
instructors. The KEGG interface contains more reference-oriented features.

The existing database web sites include graphical interfaces for the graphs
implemented as Java applets. While this approach works for desktop users, mobile
devices and tablets such as the iPad are unable to make use of them not only due
to technological constraints, but also due to the lack of touch-based interface
paradigms.

The motivation for designing and implementing tablet-based interfaces for these
databases is to help students, instructors, and researchers more effective and
less tied to their workstations. By allowing them to browse pathways with
natural gestures such as pinching and zooming, we reduce the amount of cognitive
load necessary to learn and perform intensive research tasks.
