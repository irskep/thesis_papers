iPathCase has been a successful exercise in extending the PathCase architecture
and philosophy to new devices and new users. It is one thing to design a web
interface and server backend in tandem, and another to apply the server's web
service interface to an entirely new set of applications. \mawapp and \keggapp
show that PathCase's architecture serves the purpose for which it was built: to
provide a platform for analysis and reference tools on top of multiple data
sources for pathways.

\mawapp allows browsing of a limited set of pathways in a compartment-aware
interface. It also provides the SMDA tool in an iPad environment, allowing
researchers to analyze hypotheses about their data sets with as little interface
clutter as possible.

\keggapp allows browsing of all KEGG pathways and improves on the \mawapp style
of pathway visualization. It links reactions with the external ENZYME database
and allows access to the KEGG web page corresponding to any pathway.

Both of these applications adapt the basic PathCase pathway visualization
interface concepts for a touch screen, specifically making heavy use of iOS
idioms. These adaptations provide insight into how pathway visualizations can be
improved across all forms of input and display.

Although the iPathCase applications fulfill their purpose, their design and
implementation was a challenge, and there are many possibilities for future
improvement.

\section{Technical Challenges and Possible Improvements}

The biggest challenge presented by both applications was controlling the passage
of information from the PathCase servers to the iPad application, both in how
the HTTP requests were made and how the data was translated into a form usable
by the rest of the application.

In the case of \mawapp, all parts of the \pathcasemaw database that are relevant
to \mawapp are downloaded in a single operation on a series of threads. Then,
based on the format of the data (XML, yFiles graph definition), they are parsed
and converted to Objective-C objects that are part of the \mawapp code base.
Finally, these objects are saved as Objective-C objects that can be loaded
almost instantly at any time.

This strategy is successful from a performance perspective, but code simplicity
is sacrificed to achieve it. In the future, the code that downloads the data,
parses it, and builds model objects should be refactored to separate those tasks
more explicitly. Additionally, there should be a more formal data dependency
tracking system to control the bulk download of the web service data. It is
common to have HTTP requests for web services $A$, $B$, and $C$ rely on data
from request $X$, so there should be a formalized dependency graph that tracks
this information.

In the case of \keggapp, data is requested from the \pathcasekegg server in a
much more piecemeal way. The data for each pathway is encapsulated entirely in a
single request, so there is no need for complex dependency management or even
data storage above and beyond simple HTTP request caching. The data models, HTTP
requests, and parsing are cleanly separated. \keggappp's biggest technical
issue, unlike \mawapp, is a noticeable hang when selecting a large pathway to
view. More effort should be spent moving processing to a separate thread to
allow the main user interface to remain responsive.

Another technical challenge was rendering large graphs efficiently on a device
with limited memory and processing power. Fortunately, the iOS application
frameworks made pathway rendering possible, though optimizations were required
at various stages of development.

Although the graph renderers for \mawapp and \keggapp are very similar and based
on the same code, the \keggapp version of the renderer was modified and not
reintegrated into the \mawapp code. It includes features such as the ability to
move nodes, and more aesthetically pleasing node appearances. These two
renderers should be unified, and perhaps packaged into an entirely separate
library that is not necessarily limited to pathway rendering.

Similar to graph rendering, dynamic graph layouts accounted for a considerable
amount of development time, despite the theoretical convenience of the graphviz
library. \mawapp is not able to do dynamic layouts at all, so its SMDA
hypothesis rendering capabilities are limited to single-pathway result graphs.
Now that dynamic graph layout code exists in \keggapp, it should be copied to
\mawapp and used to lay out SMDA result graphs where necessary.

\section{User Interface Challenges and Possible Improvements}

\mawappp interface is at least as effective as the Java applet version for
viewing pathways. Its most visible user interface flaw is the unusual layout of
the main screen, which would work better as a master-detail interface like
\keggappp.

The hypothesis browsing interface, while on par with the Java applet version, is
missing a way for the user to identify ``interesting'' hypotheses and export
these outside the application. One simple option is a ``star'' button on each
hypothesis and a way to email a PDF rendering of all hypotheses.

\section{Other Future Work}

iPathCase opens up many exciting possibilities for the use of tablets in
educational and research environments. The best thing we can do to make these
possibilities become realities is to perform studies and have discussions with
practicing life scientists and students. Formal and informal feedback should be
gathered from possible users to determine the most effective future
improvements. Even their simple presence in the iOS online store should provide
helpful feedback (positive or negative) that will affect the development of both
applications.

Interactive visualizations of complex biological data are no longer limited to
immobile desktop workstations or cumbersome laptops; the iPad is even less
intrusive than a physical textbook and can display information in new and
interesting ways. iPathCase brings interactive pathway visualizations, reference
materials, and even research tools to desks, meetings, and couches.
