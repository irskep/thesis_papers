Entries are listed with their name, a description of their functionality, and
an example response in a shorthand XML notation. These web services and more
are listed at \\
\href{http://nashua.case.edu/pathwaysservicekegg/PathwaysService.asmx}{http://nashua.case.edu/pathwaysservicekegg/PathwaysService.asmx}.

\subsection{Listing Pathways}
\begin{description}

    \item[AllPathways] List all pathways in the \pathcasekegg database.

    \begin{lstlisting}
<ArrayOfSoapPathway>
    <SoapPathway> (multiple)
        <Status>NoChanges</Status>
        <ID>cfb089ef-ea53-41fb-a521-e68e61de00f2</ID>
        <Name>Acridone alkaloid biosynthesis</Name>
        <Type>metabolic pathway</Type>
    \end{lstlisting}

    \item[AllPathwayGroups] List high-level groupings of pathways for user
    interface navigation purposes.

    \begin{lstlisting}
<ArrayOfSoapPathwayGroup>
    <SoapPathwayGroup> (multiple)
        <Status>NoChanges</Status>
        <Id>d5051cb1-fc0d-424b-a5a2-8b10ce7a0d78</Id>
        <Name>Amino Acid Metabolism</Name>
        <Notes>inserted on Friday, March 11, 2011</Notes>
    \end{lstlisting}

    \item[GetPathwaysByGroupId] Given a pathway grouping, list all pathways
    that group contains.

    \begin{lstlisting}
<ArrayOfString>
    <string>d92e9fb6-d2ec-4a3c-9518-12d6ef4d9fed</string> (multiple)
    \end{lstlisting}

\end{description}

\subsection{Pathway Information}
\begin{description}

    \item[GetGraphData] Given a pathway, return all data about that pathway
    necessary to render a graph except for the positions of the nodes.

    \begin{lstlisting}
<GraphData>
  <GenericProcesses>
    <GenericProcess (multiple)
          GenericProcessID=(identifier) Reversible="True"
          Name="alpha-D-Glucose 6-phosphate phosphohydrolase"
          ID=(identifier)>
      <Catalyzes>
        <Catalyze (multiple)
              OrganismGroupID=(identifier)
              GeneProductMoleculeID=(identifier)
              ECNumber="1.2.7.1" ProcessID=(identifier) />
      <Molecules>
        <Molecule (multiple)
              ProcessID=(identifier)
              Role="product" ID=(identifier) />
  <Molecules>
    <Molecule (multiple)
          EntityID=(identifier)
          IsCommon="False" Name="Thiamin diphosphate"
          ID=(identifier) />
  <Pathways>
    <Pathway (multiple)
          Name="Pyruvate metabolism" Expanded="True"
          ID=(identifier) Linking="False">
      <LinkingPathways>
        <LinkingPathway (multiple)
              ID=(identifier) Dir="out" />
          <LinkingMolecule (multiple)
            ID=(identifier) />
      <OrganismGroups>
        <OrganismGroup (multiple)
              ID=(identifier) />
      <GenericProcesses>
        <GenericProcess (multiple)
              ID=(identifier) />
    \end{lstlisting}

    \item[RetrieveLayout] Given a pathway, return the positions of its nodes
    and edges. If non exists, return an error message.

    \begin{lstlisting}
<PathwayLayout>
  <Nodes>
    <NodeLayout (multiple)
          X=(floating point number)
          Y=(floating point number)
          ID=(identifier)
          NeighboringProcessId=(identifier)
          cofactor="i" (can be 'i' for inhibitor or 'c' for cofactor) />
  <Edges>
    <EdgeLayout (multiple)
          SourceID=(identifier)
          TargetCofactor=" "
          TargetID=(identifier)
          SourceNeighboringProcessId=(identifier)
          TargetNeighboringProcessId=(identifier)
          SourceCofactor="i" (see NodeLayout[cofactor])>
      <BendPoint
        X=(floating point number)
        Y=(floating point number) />
    
    \end{lstlisting}

    \item[GetOrganismHierarchy] Return a hierarchy of all organisms in the
    database.

    \begin{lstlisting}
<OrganismHierarchy>
<OrganismGroup
        id=(identifier) commonName=""
        scientificName="Eukaryotes">
    <OrganismGroup ...> (recursively nested)
    \end{lstlisting}

    \item[GetOrganismHierarchyForAPathway] Given a pathway, return a mapping of
    reactions to the organisms they exist in.

    \begin{lstlisting}
<Pathway id=(identifier)>
    <OrganismHierarchy>
        <OrganismGroup
                id=(identifier) commonName=""
                scientificName="Eukaryotes">
            <OrganismGroup ...> (recursively nested)
                <Process (multiple)
                        id=(identifier) />
    \end{lstlisting}

\end{description}
