\section{SMDA Implementation}
\label{sect:smda_implementation}

The input interface for SMDA requires full knowledge of the SMDA database, so
\mawapp must store that database internally in some form. It must store
information about pathways, metabolites, tissues, reactions, and more.

Section \ref{sect:smda_data_model} describes the subset of the SMDA data
model that is used by \mawapp to provide the SMDA tool. Section
\ref{sect:smda_web_services_server} enumerates the web services that
SMDA provides to download these model objects into \mawapp. Section
\ref{sect:smda_web_services_client} explains how \mawapp accesses these web
services and converts their results into model objects to be used by the
application. Finally, section \ref{sect:smda_results_ui} describes the user
interface classes.

\subsection{Data Model}
\label{sect:smda_data_model}

As was mentioned at the beginning of this section, \mawapp must store a
considerable subset of the SMDA database internally. These are the objects that
are stored from that database.

The Objective-C classes representing the model objects use the \texttt{SMDA}
prefix, but the prefix is omitted in this section for clarity.

\begin{itemize}

    \item \textbf{Pathway}: Information about a pathway including name,
    identifier, and the tissue it is contained in.

    \item \textbf{Transport Process}: Information about a transport process
    including the name and identifier.

    \item \textbf{Reaction}: Information about a reaction including name,
    identifier, tissue/compartment, whether the reaction is reversible, and
    metabolite pools the reaction is associated with.

    \item \textbf{Pathway as Reaction}: An entire pathway treated as if it were
    a reaction, in order to simplify the input and output for the user. Includes
    references to a pathway and metabolite pools that are inputs and outputs.

    \item \textbf{Tissue}: Information about a tissue/compartment including
    name, identifier, parent, and compartment type.

    \item \textbf{Metabolite}: Information about a metabolite including name,
    identifier, and metabolite pools it is associated with.

    \item \textbf{Metabolite Pool}: Information about a metabolite pool
    including name, identifier, tissue/compartment, role in its associated
    reaction, and the name and identifier of its associated metabolite.

\end{itemize}

In addition to objects directly from the SMDA database, the input includes
user-defined measurements. These are the objects used to model those
measurements.

\begin{itemize}

    \item \textbf{Metabolite Measurement}: The numeric value and units of
    measurement of a single metabolite pool.

    \item \textbf{Saved Measurement}: One full set of inputs for SMDA including
    pathways, transport processes, reactions, pathways as reactions,
    tissues/compartments, metabolites, and metabolite measurements.

\end{itemize}

\subsection{Loading the Data}
\label{sect:smda_all_data}

\subsubsection{Web Services}
\label{sect:smda_web_services}

This section describes the web services that provide the data used to populate
the models described in section \ref{sect:smda_data_model}.

Some web services simply list all items of a certain type from the database. The
services of this type that \mawapp uses are \textbf{GetAllPathways},
\textbf{GetAllUnits}, \textbf{GetAllTransportProcesses},
\textbf{GetAllReactions}, \textbf{GetAllCompartments},
\textbf{GetAllMetabolites}, and \textbf{GetAllMetabolitePools}. These web
services take no arguments and always return the same result until the database
is changed.

Other web services take a pathway as an argument in order to return results that
are relevant only to that pathway. The services of this type that \mawapp uses
are \textbf{GetTissueOfPathway}, which lists tissues/compartments, and
\textbf{GetPathwayMetabolite}, which lists metabolites.

The web service \textbf{GetAllMetabolitePoolsForPathway} takes both a pathway
and as a tissue and returns metabolite pools for that pathway-tissue pair.

The web service \textbf{GetReactionMetabolite} returns associations between
reactions and metabolite pools.

By calling these web services with all possible values, the entire database can
be mirrored on the iPad.

\emph{I did not have time to enumerate all of the web services' responses. It
will take about three pages.}

\subsubsection{Storage}

Due to the hierarchical and interconnected nature of the data model, it is not
practical to dynamically invoke the web services at the time their data is
needed. Instead, an internal representation is updated in bulk at the user's
request. When they tap an ``Update Local Data'' button in the SMDA input view, a
series of asynchronously executed Objective-C blocks (\ref{sect:objc_bocks}) are
invoked. These blocks make the web requests, convert their results to the model
objects described in section \ref{sect:smda_data_model}, and serialize them to
the iPad's internal storage.

Each block represents one or more web requests, and a block may rely on the
completion of another block for the existence of data.

To avoid delays from downloading all the data the first time the application is
run, the application binary contains a snapshot of these objects taken from the
web services at the time the application was released. This way, the application
does not require internet access until the actual SMDA algorithm is run.

\subsection{Measurement Input User Interface}
\label{sect:smda_results_ui}

Saved measurements (observations) are encapsulated in the
\texttt{SMDASavedMeasurement} object. These are listed in a table view by
\texttt{SMDASavedMeasurementSetListController}. From this view, observations may
be added, removed, and renamed.

Selecting an observation from this view pushes a new view onto the navigation
stack. The new view, controlled by \texttt{SMDAMeasurementSetEditorVC}, shows
the pathways, transport processes, reactions, pathways as reactions, and
metabolite measurements contained within the observation. (See figure
\ref{fig:screenshot_smda_list}.) These data types are divided into sections of
the table. Each section has a row for each entry for the corresponding data
type, plus an ``Add'' row which can be selected to add a new item.

Selecting any row brings up a view to edit the new or existing measurement. This
view is some subclass of \texttt{SMDABasicRowEntryVC} depending on which data
type is being added/edited. It contains a list of data items such as pathways,
metabolites, or reactions, as well as additional fields such as a text box for
entering numerical values or a button for selecting the units of measurement.

When the selection has been made, the row entry view controller inserts the new
value into the observation object and returns the user to the
\texttt{SMDAMeasurementSetEditorVC}.

Tapping the ``Run SMDA'' button sends the measurements to the \pathcasemaw
server to be run through the \textbf{RunSMDA} web service discussed in section
\ref{sect:smda_results_request}.

\subsection{Getting SMDA Algorithm Results}
\label{sect:smda_results_request}

\emph{This section will describe how an SMDA web service request is formed,
submitted, read, parsed, and made into graphs. It will also explain \mawappp
inability to display results from multiple pathways due to the lack of dynamic
layouts.}

To run the SMDA algorithm, the observation is converted from Objective-C objects
to XML, sent to the \textbf{RunSMDA} web service, and the results are converted
into new Objective-C objects.

The conversion from Objective-C objects to XML is straightforward. For each data
type (pathway, transport process, reaction, pathway as reaction, metabolite
pool measurement), XML representations are concatenated. Here is Sample
Measurement 1, the example used in section \ref{sect:smda_interface_input}:

\begin{lstlisting}
<?xml version="1.0" standalone="no"?>
<SMDAipadInput>
	<Pathways>
		<Pathway ID="6"  CompartmentID="2"/>		
	</Pathways>  
	<TransportProcesses>		
	</TransportProcesses>  
	<Reactions>		
	</Reactions>  	
	<PathwayAsReactions>
	</PathwayAsReactions>  	
	<Measurements>
		<MetabolitePoolValue ID="12" Value="180"  unit="uM/ml" />
		<MetabolitePoolValue ID="31" Value="60"  unit="uM/ml" />
		<MetabolitePoolValue ID="394" Value="80"  unit="uM/ml" />
	</Measurements>
</SMDAipadInput>
\end{lstlisting}

This XML representation is passed to the \textbf{RunSMDA} web service as a
parameter to an HTTP POST request. The server processes the request and returns
an XML string representing the result graphs. A result graph consists of a base
graph consisting of reactions and metabolite pools, perhaps subdivided into
pathways and compartments, and a set of possible reaction states. Each reaction
may be active or inactive, and reversed or not.

The output of \textbf{RunSMDA} is in three parts. The first is a hierarchical
definition of the objects in the base graph, including pathways, compartments,
reactions, and metabolite pools. The second is a graphical definition of the
base graph in GML (see \ref{sect:parsing_gml}). The third is a list of lists of
reaction states. Each list of reaction states represents a possible state of the
graph, one page which the user can view.

\mawapp first builds the base graph (\texttt{PCGraphModel}) out of the GML
definition. It then annotates this graph with information from the hierarchical
data definition. Finally, it builds an \texttt{SMDAResultGraphModel} object from
this base model and the remaining information about the reaction states.

\subsubsection{Results Browsing User Interface}
\label{sect:smda_results_browsing}

The \texttt{SMDAResultGraphModel} and \texttt{PCGraphModel} produced by the
\textbf{RunSMDA} web service processing code are displayed in an
\texttt{SMDANavController} which presents the interface described in section
\ref{sect:smda_interface_output}.

The result browsing interface is a graph view with several permutations of
specific properties. A \textbf{result graph} (\texttt{SMDAResultGraphModel})
consists of a base \texttt{PCGraphModel} with the static data, plus the status
of each reaction: active or inactive, reversed or not. The result browsing
interface allows the user to page through different resuls graphs that share the
same base graph mode.

When the user taps the Forward or Back button in the \texttt{SMDANavController},
it sends a message to the \texttt{SMDAResultGraphModel} to display the result at
a given index. The \texttt{SMDAResultGraphModel} constructs a new set of node
attributes based on reaction states, applies them to the currently visible
graph, and adds highlights where attributes have changed (for example, if a node
has gone from active to inactive).
