\section{Future Work}
\label{sect:maw_future_work}

The biggest problem with the current architecture of \mawapp is its complex
system for downloading and saving the SMDA database via the web services. The
acts of making HTTP requests, parsing them, building model objects, and
serializing those objects are entangled in a handful of functions with poor
encapsulation. These concerns should be separated and a more discrete pipeline
should be constructed.

It may be beneficial to store all data model objects in a SQLite database
instead of as serialized Objective-C objects so that they can be interacted with
in a way that more closely parallels the rest of the \pathcasemaw system.

The SMDA input interface, while functional, is somewhat cumbersome to work with.
Any given input operation requires at least three taps and lots of dragging to
complete. One possibility is that \mawapp could download a spreadsheet
containing the data from Dropbox or the user's email inbox. The existing
interface could be improved by using a split view to show more data at a time
(see section \ref{sect:ipad_container_views}).

As explained in section \ref{sect:smda_results_request}, the SMDA results
browser is currently unable to deal with measurements that include more than one
pathway because it cannot generate dynamic layouts. Now that Graphviz has been
added to \keggapp, it should be possible to include that functionality in
\mawapp and provide support for measurements with more than one pathway.

The original SMDA paper by Cakmak et al \cite{smda-basic} mentions that the
number of results is exponential based on the number of measurements. It might
be helpful to add a ``star'' button to the results browsing toolbar to allow a
user to mark a hypothesis that looks particularly promising, and display all
starred hypotheses in a menu.

Another improvement would be to allow the user to export his results to some
externally readable format such as PDF. The iOS application frameworks make PDF
rendering relatively simple. Combined with the ability to ``star'' hypotheses,
this could be a good way for researchers to communicate interesting SMDA
findings.
