\subsection{Interface}
\label{sect:maw_interface}

\begin{figure}[htb]
    \center{
        % \includegraphics[width=\columnwidth]{maw/figures/screenshot_list}}
        \includegraphics[width=3in]{maw/figures/screenshot_list}}
    \caption{\label{fig:maw_screenshot_list} List of pathways on the main screen
    of PathCase MAW for iPad}
\end{figure}

\begin{figure}[hbt]
    \center{
        \includegraphics[width=3in]{maw/figures/screenshot_glycolysis}}
    \caption{\label{fig:maw_screenshot_pathway} Scrolling, zooming view of
    Glycolysis}
\end{figure}

The home screen of the MAW app displays a list of pathways that the user can
choose from to view a graph. This screen is shown in figure
\ref{fig:maw_screenshot_list}. It also contains a brief explanation of the
PathCase database and instructions for using the app. The button in the bottom
left corner activates the SMDA part of the app, described in section
\ref{sect:smda}.

After selecting a pathway, the user enters the graph view, where they can pan
and zoom across a pathway. The view for glycolysis is shown in figure
\ref{fig:maw_screenshot_pathway}.

The rectangular nodes of the graph represent processes. The rest of the nodes
represent metabolites. The edges represent relationships such as product,
substrate, cofactor, or inhibitor.

When a node is tapped, a popover appears with the full name of the object
represented by the nodes, as well as any relevant relationships or other
information.
