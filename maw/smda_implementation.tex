\section{SMDA Implementation}
\label{sect:smda_implementation}

SMDA makes full use of the \pathcasemaw data model, so \mawapp must store that
data model internally. It must store information about pathways, metabolites,
tissues, reactions, and more.

Section \ref{sect:smda_data_model} describes the subset of the \pathcasemaw data
model that is used by \mawapp to provide the SMDA tool. Section
\ref{sect:smda_web_services_server} enumerates the web services that
\pathcasemaw provides to download these model objects into \mawapp. Section
\ref{sect:smda_web_services_client} explains how \mawapp accesses these web
services and converts their results into model objects to be used by the
application.
% MORE MORE MORE

Since SMDA uses the MAW data model, it shares all model classes, with a couple
of additions. There is an \texttt{SMDASavedMeasurement} class for storing the
user's selections. There is also a subclass of \texttt{PCGraphModel} called
\texttt{SMDAResultGraphModel} which provides methods to build a model out of XML
from the web service.

The remaining classes manage the user interface. They are basic subclasses of
Cocoa objects that implement standard design patterns for iOS applications to
display the interface shown in the screenshots in section
\ref{sect:smda_interface}.

\subsection{Data Model}
\label{sect:smda_data_model}

The Objective-C classes representing the model objects use the \texttt{SMDA}
prefix, but the prefix is omitted in this section for clarity.

\begin{itemize}

    \item \textbf{Pathway}: Top level object containing all information about a
        pathway including name, identifier, 

\end{itemize}

\subsection{Web Services: Server Side}
\label{sect:smda_web_services_server}

\subsection{Web Services: Client Side}
\label{sect:smda_web_services_client}

Although the data is accessed via web services, it is needed so often and is
small enough that the entire database is distributed with the application
package in the form of serialized objects. The database can be updated whenever
the user chooses via the user interface.

Each web service corresponds roughly to one data type in the model, so those
model objects contain methods to load their data from the corresponding web
service. For example, the method \texttt{[PCPathway loadDataFromServer]}
updates all pathway objects. However, this method relies on some data for
tissues and metabolites to have been downloaded already, so the data update
process uses mutexes and asynchronous network requests to ensure that the data
is downloaded in the correct order while using parallelism as much as possible.

These web service fetching methods convert the XML response of the web service
into corresponding model objects (such as \texttt{SMDAPathway}) which are
serialized to the device's internal storage.
