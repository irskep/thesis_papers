\section{Overview}
\label{sect:overview}

The \textbf{PathCase} System is an integrated software system for storing,
managing, analyzing, and querying biological pathways at different levels of
genetic, molecular, biochemical and organismal detail \cite{pathcase-basic}. At
the computational level, PathCase allows users to visualize pathways in multiple
abstraction levels, and to pose predetermined as well as ad hoc queries using a
graphical user interface. Pathways are represented as graphs, and implemented as
a relational database. PathCase has multiple levels, with multiple tools at each
level. Currently, users can access three different PathCase system applications,
each employing a different metabolic pathways database.

This thesis includes iPad-based interfaces to two of these databases. The first
is the Metabolomics Analysis Workbench, designed for metabolomics analysis. The
PathCaseMAW database contains a fully hierarchically compartmentalized metabolic
network for humans and mice. Full compartment hierarchy refers to the
multi-tissue (e.g. liver, adipose tissue, muscle, etc) environment as well as a
complete biological compartment distinction (e.g., liver-cell, liver-cytosol,
liver-mitochondrion, etc.).

The second is a database of KEGG pathways, intended for students and
instructors. The KEGG interface contains more reference-oriented features.

The existing database web sites include graphical interfaces for the graphs
implemented as Java applets. While this approach works for desktop users, mobile
devices and tablets such as the iPad are unable to make use of them not only due
to technological constraints, but also due to the lack of touch-based interface
paradigms.

The motivation for designing and implementing tablet-based interfaces for these
databases is to help students, instructors, and researchers more effective and
less tied to their workstations. By allowing them to browse pathways with
natural gestures such as pinching and zooming, we reduce the amount of cognitive
load necessary to learn and perform intensive research and learning tasks.

\section{Implementation Notes}
\label{sect:implementation_notes}

iPad apps are implemented in the Objective-C language and the Cocoa Touch
application framework.  Cocoa provides many common facilities such as user
interface elements, network requests, asynchronous blocks of code, and touch
operations like panning and zooming. Core Animation, a subframework, provides
rendering capabilities.

Objective-C is a strict superset of C. It adds object-oriented features and
Smalltalk-style message passing to the language \cite{???}. It adds
\textbf{blocks}, which are similar to closures. It does not provide namespace
features, so most frameworks choose a two-letter prefix for all classes they
provide.

The PathCase apps use the framework for tasks such as displaying a graphical
user interface, rendering graphs, panning and zooming views, parsing XML, and
querying web services.

Cocoa follows a Model-View-Controller architecture and makes heavy use of the
delegate pattern rather than employing subclasses for custom functionality.
Interface layouts are specified in ``.xib'' files, each of which is associated
with a class. \cite{ios:application-programming-guide} Cocoa framework classes
are prefixed with \texttt{UI} or \texttt{NS}, e.g. \texttt{NSString} or
\texttt{UINavigationController}.

Core Animation is a component of Cocoa that provides rendering functionality
based on the concept of nested layers implemented using OpenGL.
\cite{ios:core-animation} Core Animation classes are prefixed with \texttt{CA},
e.g. \texttt{CALayer}.
