\section{Steady-State Metabolic Network Dynamics Analysis}
\label{sect:background_smda}

\begin{verse}
    This section is from ``SMDA for Gene Lethality Testing'' by ??? \cite{???}.
\end{verse}

SMDA is a computational metabolomics tool that captures a metabolic network and
biochemical principles in a metabolic network database. Given a set of metabolic
observations, it locates all possible activation/inactivation scenarios of the
reactions, based on the biochemical rules.

\subsection{Terminology}
\label{sect:background_smda_terminology}

The metabolic network is a connected graph $G(V,E)$ where the vertex set $V$
consists of metabolite pools and reactions, and the edge set $E$ consists of
directed edges from a vertex $u$ to vertex $v$ if (i) $u$ is a metabolite pool
that plays the role of substrate or regulator of that reaction, or (ii) $u$ is a
reaction and $v$ is a product of that reaction. The algorithm makes use of many
biochemistry principles such as \emph{Substrate}, \emph{Availability},
\emph{Product}, \emph{Inhibition}, \emph{Committed Steps}, etc. See Cakmak et
al. for details \cite{smda-basic}.

There are five discrete states a metabolite “pool” can be in: \emph{Unknown},
\emph{Unavailable}, \emph{Available}, \emph{Accumulated}, and \emph{Severely
Accumulated}. If a metabolite pool is not observed (i.e., not measured), it is
labeled as \emph{Unknown}; otherwise the algorithm compares the user-provided
observation with predefined metabolite level thresholds (originally either
obtained from HMDB3 and stored in the SMDA database, or decided manually). Let
$T_{AVAIL} (m)$, $T_{ACC} (m)$, and $T_{SAC} (m)$, $T_{AVAIL} (m) < T_{ACC} (m)
< T_{SAC} (m)$, be three metabolite level thresholds for metabolite $m$. Also
let the observed value for metabolite $m$ be $Obs(m)$. If $Obs(m) < T_{AVAIL}
(m)$ then we mark the metabolite pool m as \emph{Unavailable}; if $T_{AVAIL} (m)
\leq Obs(m) < T_{ACC}(m)$ then we mark $m$ as \emph{Available}; if $T_{ACC} (m)
\leq Obs(m) < T_{SAC} (m)$ then we mark $m$ as \emph{Accumulated}; and, finally,
if $T_{SAC}(m) \leq Obs(m)$ then we mark $m$ as \emph{Severely Accumulated}.

There are three discrete states for a reaction, namely, \emph{Unknown},
\emph{Active} and \emph{Inactive}. Initially, all reactions are labeled as
\emph{Unknown}.  Each reaction has a set of conditions for it to be
\emph{Active}. For instance, for reaction $r$ to be \emph{Active}, substrate $s$
has to be labeled as \emph{Available}, and product $p$ has to be labeled as
\emph{Available} or \emph{Accumulated}.

A basic assumption of the SMDA algorithm is that the metabolic profile
(observations) are obtained when the organism is in a steady state. That is, for
a time interval $T$, (i) the production rate of each metabolite pool in the
network is equal to its consumption rate, and (ii) the metabolite pool labels
stay constant. This assumption corresponds to the homoeostasis of the organism.

The SMDA algorithm makes use of two main concepts: \emph{Activation/Inactivation
Graph} ($G_{AI}$) and $G_{AI}$ Group. A $G_{AI}$ is a connected sub-graph of the
metabolic sub-network, where (i) each metabolite pool is assigned a label other
than \emph{Unknown} (\emph{Unavailable}, \emph{Available}, \emph{Accumulated},
or \emph{Severely Accumulated}) and (ii) each reaction is assigned a label other
than \emph{Unknown} (\emph{Active} or \emph{Inactive}). A $G_{AI}$ group is a set
of $G_{AI}$s, where (a) all $G_{AI}$s share the same set of reactions and metabolite
pools, and, (iii) any two $G_{AI}$‘s of the $G_{AI}$ group differ by at least one
metabolite pool label assignment.  Put another way, a $G_{AI}$ group represents all
possible activation/inactivation scenarios within a selected sub-graph of the
query network. An alternative output to $G_{AI}$ graphs is flow-graphs where a
flow-graph is a $G_{AI}$ graph without metabolite pool labels, and a single
flow-graph captures multiple $G_{AI}$ graphs.
