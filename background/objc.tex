\section{Objective-C}
\label{sect:objc}

iOS applications, and therefore iPad applications, are written in the
Objective-C programming language.

\begin{quote}

    The Objective-C language is a simple computer language designed to enable
    sophisticated object-oriented programming. Objective-C is defined as a small
    but powerful set of extensions to the standard ANSI C language. Its
    additions to C are mostly based on Smalltalk, one of the first
    object-oriented programming languages. Objective-C is designed to give C
    full object-oriented programming capabilities, and to do so in a simple and
    straightforward way.

    \raggedleft \em The Objective-C Programming Language \cite{objc:manual}

\end{quote}

Objective-C is similar in some ways to other object-oriented languages like Java
and C++ in that it has classes, inheritance, polymorphism, and interfaces
(called protocols in Objective-C). Most of its differences arise from its
Smalltalk origins and its intentional compatibility with regular C code.

Objective-C is a dynamic programming language: types and methods are determined
(and changed) at runtime. Its dynamic behavior is handled by the \emph{runtime
library}, which is a C library that operates on the language's underlying
primitives to achive dynamic behavior. These primitives, including classes and
objects, are represented as C structs, unions, arrays, etc. under the hood,
allowing programmers to mix performance-sensitive low-level code with readable,
high-level code.

\subsubsection{Objects, and Messages}
\label{sect:objc_objects}

An Objective-C class is defined in a \emph{header file} and an
\emph{implementation file}. Like C, the header file contains constants, class
attributes, and function and method prototypes. The implementation file contains
their implementations, as well as any private information. For example,
\texttt{Rectangle.h} might look like this:

\begin{objc}
@interface Rectangle : NSObject {
    @private
    float width, height;
    float x, y;
}
- (float)width;
- (float)height;
- (void)setOriginX:(float)newX y:(float)newY;
- (void)setSizeWidth:(float)newWidth height:(float)newHeight;
@end
\end{objc}

\texttt{width}, \texttt{height}, \texttt{x}, and \texttt{y} are instance
variables. They can be any C data type, or pointers to Objective-C objects. They
are private by default, accessible via getters and setters. There is syntactical
sugar to define the instance variable, getter, and setter in a single line:

\begin{objc}
@interface Rectangle : NSObject {
    // ...
}
@property (nonatomic, assign) float width;
// ...
@end
\end{objc}

The companion file to \texttt{Rectangle.h}, \texttt{Rectangle.m}, would contain
implementations of the functions in \texttt{Rectangle.h}, as well as any private
functions and information.

Rather than accessing an object's methods by function calls bound at compile
time, Objective-C uses \emph{message passing}, in which the target of a message
is determined at runtime. Message syntax differs from traditional function
syntax in that each parameter is both position-dependent (like C) and has a name
that the programmer must use when passing a message. These are all valid
messages to the \texttt{Rectangle} class defined above:

\begin{objc}
// Message without arguments
float width = [myRectangle width];
// Message with arguments, one derived from a C function
[myRectangle setOriginWithX:sinf(10.0f) y:20.0f];
\end{objc}

However, this would not be valid:

\begin{objc}
[myRectangle y:20.0f setOriginWithX:10.0f];
\end{objc}

Other than these important differences, Objective-C may be understood to behave
like other object-oriented languages such as C++ with respect to its type
system, albeit with different syntax and terminology.

\subsubsection{Blocks}
\label{sect:objc_blocks}

Blocks are a kind of anonymous function. They are similar to C functions, but
they can also contain references to stack and heap memory outside their scope,
and are managed in memory like Objective-C objects \cite{objc:blocks}. They are often used for
event callbacks.

Here is a simple usage of blocks:

\begin{objc}
int x = 123;
 
void (^printXAndY)(int) = ^(int y) {
    printf("%d %d\n", x, y);
};

printXAndY(456); // prints: 123 456
\end{objc}

\subsubsection{Namespaces}
\label{sect:objc_namespaces}

Objective-C does not have multiple namespaces. All public objects, functions,
and methods share the global namespace, just like in C. To avoid collisions, the
Objective-C convention is to use two-letter prefixes on all classes.

The Cocoa base classes and data structures use the \texttt{NS} prefix after
Cocoa's NextSTEP roots, e.g. \texttt{NSObject} and \texttt{NSString}. iOS user
interface classes use \texttt{UI}. Each framework chooses a (supposedly) unique
prefix.

\subsubsection{Memory Management}
\label{sect:objc_memory}

As a language meant to interact closely with C, early versions of Objective-C
offered only a primitive form of memory management: manual reference counting.
Under this system, each object has a \emph{reference count}. An object's
reference count is incremented (\emph{retained}) when a block of code (or
another object) takes ownership of it, and decremented (\emph{released}) when
that block of code or object no longer needs it. Each object releases all of its
instance variables when its retain count becomes zero.

An object $A$ may be \emph{autoreleased} if the function $X$ \emph{created} the
object but does not wish to \emph{own} it. For example, $X$ may be a getter, and
$X$'s caller will immediately take ownership of the object by calling
\texttt{retain} on $A$. To get around this problem of temporary ownership, an
autoreleased object's reference count is incremented, but the object is put in a
global set of autoreleased objects to be released (``draining the autorelease
pool'') at some convenient point in the future.

These rules have confused programmers of all skill levels. To decrease program
complexity, two new forms of memory management have been added to the language
since its creation. One is garbage collection, not available on iOS, and the
other is \emph{automatic reference counting}.

Automatic reference counting (ARC) is similar to garbage collection in that most
of the burden of tracking object ownership is lifted from the programmer. But
ARC, unlike most garbage collection schemes, is a compile-time step; the
compiler analyzes the code and inserts \texttt{retain}, \texttt{release}, and
\texttt{autorelease} statements automatically.
