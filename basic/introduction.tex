\section{Overview}
\label{sect:overview}

The \textbf{PathCase} System \cite{pathcase-basic} is an integrated software
system for storing, managing, analyzing, and querying biological pathways at
different levels of genetic, molecular, biochemical and organismal detail
(\ref{sect:pathcase_overview}). At the computational level, PathCase allows
users to visualize pathways in multiple abstraction levels, and to pose
predetermined as well as ad hoc queries using a graphical user interface.
Pathways are represented as graphs, and implemented as a relational database.
PathCase has multiple levels, with multiple tools at each level. Currently,
users can access three different PathCase system applications, each employing a
different database of metabolic pathways.

This thesis presents iPad applications corresponding to two of these databases,
each with a slightly different focus. The first application is \keggapp, which
uses the web services of \pathcasekegg to present interactive pathway
visualizations of pathways from KEGG (Kyoto Encyclopedia of Genes and Genomes).
For more information about KEGG, see section \ref{sect:background_kegg}.
\keggapp makes use of multiple data sources, including \pathcasekegg, to enrich
pathway visualizations as much as possible.

The second application is \mawapp, designed for metabolomics analysis and
browsing of the compartment-aware \pathcasemaw pathways database. The
\pathcasemaw database contains a fully hierarchically compartmentalized
metabolic network for humans and mice. Full compartment hierarchy refers to the
multi-tissue (liver, adipose tissue, muscle, etc) environment as well as a
complete biological compartment distinction (liver-cell, liver-cytosol,
liver-mitochondrion, etc.). \mawapp provides fewer reference features than
\keggapp, but it includes a fully functional version of the SMDA (Steady-State
Metabolic Network Dynamics Analysis) tool.

Many of the features of these iPad applications are present in the web-based
\pathcasemaw and \pathcasekegg tools. For example, users can already browse
pathways and interact with visualizations of biological pathway graphs. But
there are several motivations for designing new tablet-based applications.

From the perspective of a PathCase user, the arrival of PathCase applications on
mobile devices lifts many restrictions on the settings in which the system may
be used. An iPad takes up less space than a textbook or laptop while offering
more effective reference and research materials due to its simple but
highly interactive interface. Even the simple act of scrolling content in two
dimensions makes tasks such as scanning over a pathway visualization much
simpler.

Perhaps equally important is the opportunity to test the modular nature of the
PathCase architecture. As section \ref{sect:pathcase_architecture} will explain,
the PathCase system is designed to support multiple clients to the data store,
only one of which is the web-based interface. \keggapp and \mawapp rely on the
public web services of \pathcasekegg and \pathcasemaw for all of the data they
display. By implementing entirely new applications, we identify important gaps
in the information these web services provide.

\section{Document Structure}

Chapter \ref{ch:background} contains background information on all topics
necessary to understand the design and implementation of \keggapp and \mawapp.
The first half, sections \ref{sect:metabolic_pathways} through
\ref{sect:pathcase}, describe the relevant biology and the systems that \keggapp
and \mawapp rely on. This information is necessary to understand the design. The
second half, sections \ref{sect:ipad} through \ref{sect:cocoa}, describe the
technology used to develope iPad applications. This information is necessary to
understand the implementation.

Chapter \ref{ch:maw_smda} describes \mawapp in depth. It includes the design and
implementation of \mawapp itself, modifications that were made to \pathcasemaw
to support it, its differences from similar functionality in the \pathcasemaw
web interface, and possible future work. \mawapp is covered before \keggapp
in this paper because it was developed first chronologically.

Chapter \ref{ch:kegg} describes \keggapp in similar detail to \mawapp.

Chapter \ref{ch:related_work} is an overview of similar work in the areas of
graph visualization, touch-based gestures, and biological pathway visualization.
Chapter \ref{ch:conclusion} concludes the paper and discusses the challenges,
successes, failures, and possible future directions for the project as a whole.

Appendix \ref{ch:kegg_manual} is the user manual for \keggapp. Appendix
\ref{ch:smda_code_docs} is the documentation for all classes in \mawapp.
Appendix \ref{ch:kegg_code_docs} is the documentation for all classes in
\keggapp.
