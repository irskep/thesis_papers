\section{SMDA Implementation}
\label{sect:smda_implementation}

The input interface for SMDA requires full knowledge of the SMDA database, so
\mawapp must store that database internally in some form.  It must store
information about pathways, metabolites, tissues, reactions, and more.

Section \ref{sect:smda_data_model} describes the subset of the \pathcasemaw data
model that is used by \mawapp to provide the SMDA tool. Section
\ref{sect:smda_web_services_server} enumerates the web services that
\pathcasemaw provides to download these model objects into \mawapp. Section
\ref{sect:smda_web_services_client} explains how \mawapp accesses these web
services and converts their results into model objects to be used by the
application.
% MORE MORE MORE

\subsection{Data Model}
\label{sect:smda_data_model}

As was mentioned at the beginning of this section, \mawapp must store a
considerable subset of the SMDA database internally. These are the objects that
are stored from that database.

The Objective-C classes representing the model objects use the \texttt{SMDA}
prefix, but the prefix is omitted in this section for clarity.

\begin{itemize}

    \item \textbf{Pathway}: Information about a pathway including name,
    identifier, and the tissue it is contained in.

    \item \textbf{Transport Process}: Information about a transport process
    including the name and identifier.

    \item \textbf{Reaction}: Information about a reaction including name,
    identifier, tissue/compartment, whether the reaction is reversible, and
    metabolite pools the reaction is associated with.

    \item \textbf{Pathway as Reaction}: An entire pathway treated as if it were
    a reaction, in order to simplify the input and output for the user. Includes
    references to a pathway and metabolite pools that are inputs and outputs.

    \item \textbf{Tissue}: Information about a tissue/compartment including
    name, identifier, parent, and compartment type.

    \item \textbf{Metabolite}: Information about a metabolite including name,
    identifier, and metabolite pools it is associated with.

    \item \textbf{Metabolite Pool}: Information about a metabolite pool
    including name, identifier, tissue/compartment, role in its associated
    reaction, and the name and identifier of its associated metabolite.

\end{itemize}

In addition to objects directly from the SMDA database, the input includes
user-defined measurements. These are the objects used to model those
measurements.

\begin{itemize}

    \item \textbf{Metabolite Measurement}: The numeric value and units of
    measurement of a single metabolite pool.

    \item \textbf{Saved Measurement}: One full set of inputs for SMDA including
    pathways, transport processes, reactions, pathways as reactions,
    tissues/compartments, metabolites, and metabolite measurements.

\end{itemize}

\subsection{Web Services: Server Side}
\label{sect:smda_web_services_server}

This section describes the web services that provide the data used to populate
the models described in section \ref{sect:smda_data_model}.

\begin{itemize}

    \item \textbf{GetAllPathways}
    \item \textbf{GetTissueOfPathway}
    \item \textbf{GetAllMetabolitePoolsForPathway}
    \item \textbf{GetPathwayMetabolite}
    \item \textbf{GetAllUnits}
    \item \textbf{GetAllTransportProcesses}
    \item \textbf{GetAllReactions}
    \item \textbf{GetReactionMetabolite}
    \item \textbf{GetAllCompartments}
    \item \textbf{GetAllMetabolites}
    \item \textbf{GetAllMetabolitePools}

\end{itemize}

\subsection{Web Services: Client Side}
\label{sect:smda_web_services_client}

Although the data is accessed via web services, it is needed so often and is
small enough that the entire database is distributed with the application
package in the form of serialized objects. The database can be updated whenever
the user chooses via the user interface.

Each web service corresponds roughly to one data type in the model, so those
model objects contain methods to load their data from the corresponding web
service. For example, the method \texttt{[PCPathway loadDataFromServer]}
updates all pathway objects. However, this method relies on some data for
tissues and metabolites to have been downloaded already, so the data update
process uses mutexes and asynchronous network requests to ensure that the data
is downloaded in the correct order while using parallelism as much as possible.

These web service fetching methods convert the XML response of the web service
into corresponding model objects (such as \texttt{SMDAPathway}) which are
serialized to the device's internal storage.
